


{\color{Blue}{\subsection{Purpose}}}
{\setlength{\parskip}{0.5cm}
\raggedright

This Design Document will provide an overall definition of the main system components and the relative interactions, an in-depth discussion about the chosen architectural patterns and the plan for integration, verification and validation steps. The audience for this artifact is tighter than the one of the RASD due to the more technical language. In particular, it is oriented to all the people involved in engineering the software-to-be. In any case, in order to facilitate the comprehension of this document a previous RASD consultation is strongly suggested.

{\color{Blue}{\subsection{Scope}}}

As already explained in the previous RASD, Data4Help and AutomatedSOS are physically separated applications, since even if they handle the same set of data types, they concern with different purposes. The first one represents a privacy-guaranteed service to manage third party requests of access to user-provided health data, the second is a service conceived to ensure first assistance whenever dangerous conditions for registered users are detected. The discussion at this point should be focused specifically on the architecture behind these two services, analyzing the way both applications will collect health data from the monitoring devices, how TrackMe servers will manage such data and more. To this end, since TrackMe wants to not depend strictly on external cloud infrastructures for commercial and data-security reasons, it decided to host  its own server farm completely. For the moment the system will be centralized, but easily distributable for future needs, according to the growth factor of the customer number. This in order to form a private cloud infrastructure built on top of a distributed database system. 
%Unfortunately, the deadline prevents a detailed analysis of this solution for further versions of the project.\\



{\color{Blue}{\subsection{Definitions, Acronyms, Abbreviations}}}
{\color{Blue}{\subsubsection{Definitions}}}
\textbf{User:} both Data4Help and AutomatedSOS users.\\
\textbf{Third Party:} more frequently are companies that looks for health information for their market research, but they could also be facilities like hospitals or other activities interested in the health field.  \\ 
\textbf{Anomaly condition:} alteration of health monitored parameters that causes an alert to the user.\\
\textbf{Emergency condition:} alteration of the health monitored parameters that directly triggers the sent of an ambulance.\\
\textbf{Supervisor user:} every user that has access to another user information and location. \\
\textbf{Sensor/Monitoring Device:} all the health sensor devices that can be registered in the Data4Help system.\\
\textbf{Supervised User:} an AutomatedSOS user who is monitored by the application and has one or more Supervisors. \\
\textbf{ASOS Monitor Mode:} flag that tells the system if the user must be monitored or not, it can be changed from client-side application.\\
\textbf{Emergency Resource Manager:} is a system already implemented in all the hospitals subscribed to the AutomatedSOS service, it handles emergency calls and AutomatedSOS notifications. For each case manages the resources required by the emergency. \\
\textbf{RescueSquad/Resource:} general term used to refer a registered group of rescuers provided with the equipment required for the specific type of emergency. 


{\color{Blue}\subsubsection{Acronyms}}
\textbf{ASOS:} AutomatedSOS\\
\textbf{D4H:} Data4Help\\
\textbf{API:} Application Programming Interface\\
\textbf{RASD:} Requirement Analysis Specification Document \\
\textbf{HW:} Hardware\\
\textbf{BLE:} Bluetooth Low Energy\\
\textbf{TLS:} Transport Layer Security\\
\textbf{HTTPS:} HTTP over TLS\\
\textbf{ERM:} Emergency Resource Manager\\
\textbf{TP:} Third Party


{\color{Blue}\subsubsection{Abbreviations}}
\textbf{[Gn]:} n-th Goal\\
\textbf{[Rn]:} n-th Requirement


{\color{Blue}{\subsection{Revision History}}}
\textbf{Revision 1:}\begin{itemize}
	\item Minor changes
\end{itemize}

{\color{Blue}{\subsection{Reference Documents}}}
\begin{itemize}
	\item Mandatory Project Assignment AY 2018-2019
	\item\textit{Requirements Analysis and Specification Document}, version 2.0
\end{itemize}

{\color{Blue}{\subsection{Document Structure}}}

{\color{Blue}{\subsubsection{Chapter 1}}}
In the first chapter is given a general presentation of the aim of this document. Moreover are given other basic information in order to read easily the entire work, like the dictionary for example.
%\paragraph{}

{\color{Blue}{\subsubsection{Chapter 2}}}
In the second chapter the entire architecture is defined, starting from an high-level view of the system.



{\color{Blue}{\subsubsection{Chapter 3}}}
In chapter 3, the GUI system is clearly represented by the UX diagrams. A short description about the adopted notation is provided in the beginning of the relative section.


{\color{Blue}{\subsubsection{Chapter 4}}}
This chapter is dedicated to explain how the architecture components are mapped with the requirements defined in the previous artifact.


{\color{Blue}{\subsubsection{Chapter 5}}}
This part is dedicated to discuss about implementation, integration and testing strategies and plans for the further stages. 

}